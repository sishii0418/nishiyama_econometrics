% Options for packages loaded elsewhere
\PassOptionsToPackage{unicode}{hyperref}
\PassOptionsToPackage{hyphens}{url}
%
\documentclass[
]{book}
\usepackage{amsmath,amssymb}
\usepackage{iftex}
\ifPDFTeX
  \usepackage[T1]{fontenc}
  \usepackage[utf8]{inputenc}
  \usepackage{textcomp} % provide euro and other symbols
\else % if luatex or xetex
  \usepackage{unicode-math} % this also loads fontspec
  \defaultfontfeatures{Scale=MatchLowercase}
  \defaultfontfeatures[\rmfamily]{Ligatures=TeX,Scale=1}
\fi
\usepackage{lmodern}
\ifPDFTeX\else
  % xetex/luatex font selection
\fi
% Use upquote if available, for straight quotes in verbatim environments
\IfFileExists{upquote.sty}{\usepackage{upquote}}{}
\IfFileExists{microtype.sty}{% use microtype if available
  \usepackage[]{microtype}
  \UseMicrotypeSet[protrusion]{basicmath} % disable protrusion for tt fonts
}{}
\makeatletter
\@ifundefined{KOMAClassName}{% if non-KOMA class
  \IfFileExists{parskip.sty}{%
    \usepackage{parskip}
  }{% else
    \setlength{\parindent}{0pt}
    \setlength{\parskip}{6pt plus 2pt minus 1pt}}
}{% if KOMA class
  \KOMAoptions{parskip=half}}
\makeatother
\usepackage{xcolor}
\usepackage{color}
\usepackage{fancyvrb}
\newcommand{\VerbBar}{|}
\newcommand{\VERB}{\Verb[commandchars=\\\{\}]}
\DefineVerbatimEnvironment{Highlighting}{Verbatim}{commandchars=\\\{\}}
% Add ',fontsize=\small' for more characters per line
\usepackage{framed}
\definecolor{shadecolor}{RGB}{248,248,248}
\newenvironment{Shaded}{\begin{snugshade}}{\end{snugshade}}
\newcommand{\AlertTok}[1]{\textcolor[rgb]{0.94,0.16,0.16}{#1}}
\newcommand{\AnnotationTok}[1]{\textcolor[rgb]{0.56,0.35,0.01}{\textbf{\textit{#1}}}}
\newcommand{\AttributeTok}[1]{\textcolor[rgb]{0.13,0.29,0.53}{#1}}
\newcommand{\BaseNTok}[1]{\textcolor[rgb]{0.00,0.00,0.81}{#1}}
\newcommand{\BuiltInTok}[1]{#1}
\newcommand{\CharTok}[1]{\textcolor[rgb]{0.31,0.60,0.02}{#1}}
\newcommand{\CommentTok}[1]{\textcolor[rgb]{0.56,0.35,0.01}{\textit{#1}}}
\newcommand{\CommentVarTok}[1]{\textcolor[rgb]{0.56,0.35,0.01}{\textbf{\textit{#1}}}}
\newcommand{\ConstantTok}[1]{\textcolor[rgb]{0.56,0.35,0.01}{#1}}
\newcommand{\ControlFlowTok}[1]{\textcolor[rgb]{0.13,0.29,0.53}{\textbf{#1}}}
\newcommand{\DataTypeTok}[1]{\textcolor[rgb]{0.13,0.29,0.53}{#1}}
\newcommand{\DecValTok}[1]{\textcolor[rgb]{0.00,0.00,0.81}{#1}}
\newcommand{\DocumentationTok}[1]{\textcolor[rgb]{0.56,0.35,0.01}{\textbf{\textit{#1}}}}
\newcommand{\ErrorTok}[1]{\textcolor[rgb]{0.64,0.00,0.00}{\textbf{#1}}}
\newcommand{\ExtensionTok}[1]{#1}
\newcommand{\FloatTok}[1]{\textcolor[rgb]{0.00,0.00,0.81}{#1}}
\newcommand{\FunctionTok}[1]{\textcolor[rgb]{0.13,0.29,0.53}{\textbf{#1}}}
\newcommand{\ImportTok}[1]{#1}
\newcommand{\InformationTok}[1]{\textcolor[rgb]{0.56,0.35,0.01}{\textbf{\textit{#1}}}}
\newcommand{\KeywordTok}[1]{\textcolor[rgb]{0.13,0.29,0.53}{\textbf{#1}}}
\newcommand{\NormalTok}[1]{#1}
\newcommand{\OperatorTok}[1]{\textcolor[rgb]{0.81,0.36,0.00}{\textbf{#1}}}
\newcommand{\OtherTok}[1]{\textcolor[rgb]{0.56,0.35,0.01}{#1}}
\newcommand{\PreprocessorTok}[1]{\textcolor[rgb]{0.56,0.35,0.01}{\textit{#1}}}
\newcommand{\RegionMarkerTok}[1]{#1}
\newcommand{\SpecialCharTok}[1]{\textcolor[rgb]{0.81,0.36,0.00}{\textbf{#1}}}
\newcommand{\SpecialStringTok}[1]{\textcolor[rgb]{0.31,0.60,0.02}{#1}}
\newcommand{\StringTok}[1]{\textcolor[rgb]{0.31,0.60,0.02}{#1}}
\newcommand{\VariableTok}[1]{\textcolor[rgb]{0.00,0.00,0.00}{#1}}
\newcommand{\VerbatimStringTok}[1]{\textcolor[rgb]{0.31,0.60,0.02}{#1}}
\newcommand{\WarningTok}[1]{\textcolor[rgb]{0.56,0.35,0.01}{\textbf{\textit{#1}}}}
\usepackage{longtable,booktabs,array}
\usepackage{calc} % for calculating minipage widths
% Correct order of tables after \paragraph or \subparagraph
\usepackage{etoolbox}
\makeatletter
\patchcmd\longtable{\par}{\if@noskipsec\mbox{}\fi\par}{}{}
\makeatother
% Allow footnotes in longtable head/foot
\IfFileExists{footnotehyper.sty}{\usepackage{footnotehyper}}{\usepackage{footnote}}
\makesavenoteenv{longtable}
\usepackage{graphicx}
\makeatletter
\def\maxwidth{\ifdim\Gin@nat@width>\linewidth\linewidth\else\Gin@nat@width\fi}
\def\maxheight{\ifdim\Gin@nat@height>\textheight\textheight\else\Gin@nat@height\fi}
\makeatother
% Scale images if necessary, so that they will not overflow the page
% margins by default, and it is still possible to overwrite the defaults
% using explicit options in \includegraphics[width, height, ...]{}
\setkeys{Gin}{width=\maxwidth,height=\maxheight,keepaspectratio}
% Set default figure placement to htbp
\makeatletter
\def\fps@figure{htbp}
\makeatother
\setlength{\emergencystretch}{3em} % prevent overfull lines
\providecommand{\tightlist}{%
  \setlength{\itemsep}{0pt}\setlength{\parskip}{0pt}}
\setcounter{secnumdepth}{5}
\usepackage{booktabs}
\ifLuaTeX
  \usepackage{selnolig}  % disable illegal ligatures
\fi
\usepackage[]{natbib}
\bibliographystyle{plainnat}
\IfFileExists{bookmark.sty}{\usepackage{bookmark}}{\usepackage{hyperref}}
\IfFileExists{xurl.sty}{\usepackage{xurl}}{} % add URL line breaks if available
\urlstyle{same}
\hypersetup{
  pdftitle={西山ほか(2019)『計量経済学』有斐閣の練習問題解答とRでの再現},
  pdfauthor={石井 俊輔},
  hidelinks,
  pdfcreator={LaTeX via pandoc}}

\title{西山ほか(2019)『計量経済学』有斐閣の練習問題解答とRでの再現}
\author{石井 俊輔}
\date{2024-04-28}

\begin{document}
\maketitle

{
\setcounter{tocdepth}{1}
\tableofcontents
}
\hypertarget{ux306fux3058ux3081ux306b}{%
\chapter*{はじめに}\label{ux306fux3058ux3081ux306b}}
\addcontentsline{toc}{chapter}{はじめに}

西山ほか(2019)『計量経済学』有斐閣 (\href{https://www.yuhikaku.co.jp/books/detail/9784641053854}{出版社リンク}) の練習問題解答とRでの再現です.
必要なRパッケージをインストール:

\begin{Shaded}
\begin{Highlighting}[]
\FunctionTok{install.packages}\NormalTok{(}\StringTok{"tidyverse"}\NormalTok{)}
\DocumentationTok{\#\# Installing package into \textquotesingle{}/home/shunsuke/R/x86\_64{-}pc{-}linux{-}gnu{-}library/4.3\textquotesingle{}}
\DocumentationTok{\#\# (as \textquotesingle{}lib\textquotesingle{} is unspecified)}
\FunctionTok{install.packages}\NormalTok{(}\StringTok{"openxlsx"}\NormalTok{)}
\DocumentationTok{\#\# Installing package into \textquotesingle{}/home/shunsuke/R/x86\_64{-}pc{-}linux{-}gnu{-}library/4.3\textquotesingle{}}
\DocumentationTok{\#\# (as \textquotesingle{}lib\textquotesingle{} is unspecified)}
\FunctionTok{install.packages}\NormalTok{(}\StringTok{"haven"}\NormalTok{)}
\DocumentationTok{\#\# Installing package into \textquotesingle{}/home/shunsuke/R/x86\_64{-}pc{-}linux{-}gnu{-}library/4.3\textquotesingle{}}
\DocumentationTok{\#\# (as \textquotesingle{}lib\textquotesingle{} is unspecified)}
\FunctionTok{install.packages}\NormalTok{(}\StringTok{"wooldridge"}\NormalTok{)}
\DocumentationTok{\#\# Installing package into \textquotesingle{}/home/shunsuke/R/x86\_64{-}pc{-}linux{-}gnu{-}library/4.3\textquotesingle{}}
\DocumentationTok{\#\# (as \textquotesingle{}lib\textquotesingle{} is unspecified)}
\FunctionTok{install.packages}\NormalTok{(}\StringTok{"fixest"}\NormalTok{)}
\DocumentationTok{\#\# Installing package into \textquotesingle{}/home/shunsuke/R/x86\_64{-}pc{-}linux{-}gnu{-}library/4.3\textquotesingle{}}
\DocumentationTok{\#\# (as \textquotesingle{}lib\textquotesingle{} is unspecified)}
\end{Highlighting}
\end{Shaded}

\hypertarget{usage}{%
\section{Usage}\label{usage}}

Each \textbf{bookdown} chapter is an .Rmd file, and each .Rmd file can contain one (and only one) chapter. A chapter \emph{must} start with a first-level heading: \texttt{\#\ A\ good\ chapter}, and can contain one (and only one) first-level heading.

Use second-level and higher headings within chapters like: \texttt{\#\#\ A\ short\ section} or \texttt{\#\#\#\ An\ even\ shorter\ section}.

The \texttt{index.Rmd} file is required, and is also your first book chapter. It will be the homepage when you render the book.

\hypertarget{render-book}{%
\section{Render book}\label{render-book}}

You can render the HTML version of this example book without changing anything:

\begin{enumerate}
\def\labelenumi{\arabic{enumi}.}
\item
  Find the \textbf{Build} pane in the RStudio IDE, and
\item
  Click on \textbf{Build Book}, then select your output format, or select ``All formats'' if you'd like to use multiple formats from the same book source files.
\end{enumerate}

Or build the book from the R console:

\begin{Shaded}
\begin{Highlighting}[]
\NormalTok{bookdown}\SpecialCharTok{::}\FunctionTok{render\_book}\NormalTok{()}
\end{Highlighting}
\end{Shaded}

To render this example to PDF as a \texttt{bookdown::pdf\_book}, you'll need to install XeLaTeX. You are recommended to install TinyTeX (which includes XeLaTeX): \url{https://yihui.org/tinytex/}.

\hypertarget{preview-book}{%
\section{Preview book}\label{preview-book}}

As you work, you may start a local server to live preview this HTML book. This preview will update as you edit the book when you save individual .Rmd files. You can start the server in a work session by using the RStudio add-in ``Preview book'', or from the R console:

\begin{Shaded}
\begin{Highlighting}[]
\NormalTok{bookdown}\SpecialCharTok{::}\FunctionTok{serve\_book}\NormalTok{()}
\end{Highlighting}
\end{Shaded}

\hypertarget{ch2}{%
\chapter*{第2章 データの整理と確率変数の基礎}\label{ch2}}
\addcontentsline{toc}{chapter}{第2章 データの整理と確率変数の基礎}

先に\href{https://www.yuhikaku.co.jp/books/detail/9784641053854}{出版社サイト}よりデータをダウンロードする.

\begin{Shaded}
\begin{Highlighting}[]
\CommentTok{\# サポートファイルへのリンク}
\NormalTok{curl }\OtherTok{\textless{}{-}} \StringTok{"https://www.yuhikaku.co.jp/static\_files/05385\_support02.zip"}
\CommentTok{\# ダウンロード保存用フォルダが存在しない場合, 作成}
\ControlFlowTok{if}\NormalTok{(}\SpecialCharTok{!}\FunctionTok{dir.exists}\NormalTok{(}\StringTok{"downloads"}\NormalTok{))\{}
    \FunctionTok{dir.create}\NormalTok{(}\StringTok{"downloads"}\NormalTok{)}
\NormalTok{\}}
\NormalTok{cdestfile }\OtherTok{\textless{}{-}} \StringTok{"downloads/support02.zip"}
\FunctionTok{download.file}\NormalTok{(curl, cdestfile)}
\CommentTok{\# データ保存用フォルダが存在しない場合, 作成}
\ControlFlowTok{if}\NormalTok{(}\SpecialCharTok{!}\FunctionTok{dir.exists}\NormalTok{(}\StringTok{"data"}\NormalTok{))\{}
    \FunctionTok{dir.create}\NormalTok{(}\StringTok{"data"}\NormalTok{)}
\NormalTok{\}}
\CommentTok{\# WSL上のRで解凍すると文字化けするので、Linuxのコマンドを外部呼び出し}
\CommentTok{\# Windowsの場合は別途コマンドを用いる.}
\ControlFlowTok{if}\NormalTok{(.Platform}\SpecialCharTok{$}\NormalTok{OS.type }\SpecialCharTok{==} \StringTok{"unix"}\NormalTok{) \{}
    \FunctionTok{system}\NormalTok{(}\FunctionTok{sprintf}\NormalTok{(}\StringTok{\textquotesingle{}unzip {-}n {-}Ocp932 \%s {-}d \%s\textquotesingle{}}\NormalTok{, }\StringTok{"downloads/support02.zip"}\NormalTok{, }\StringTok{"./data"}\NormalTok{))}
\NormalTok{\} }\ControlFlowTok{else}\NormalTok{ \{}
    \FunctionTok{print}\NormalTok{(}\StringTok{"Windowsで解凍するコマンドを別途追加せよ."}\NormalTok{)}
\NormalTok{\}}
\end{Highlighting}
\end{Shaded}

必要なライブラリを読み込む.

\begin{Shaded}
\begin{Highlighting}[]
\FunctionTok{library}\NormalTok{(tidyverse)}
\DocumentationTok{\#\# {-}{-} Attaching core tidyverse packages {-}{-}{-}{-}{-}{-}{-}{-}{-}{-}{-}{-}{-}{-}{-}{-}{-}{-}{-}{-}{-}{-}{-}{-} tidyverse 2.0.0 {-}{-}}
\DocumentationTok{\#\# v dplyr     1.1.4     v readr     2.1.5}
\DocumentationTok{\#\# v forcats   1.0.0     v stringr   1.5.1}
\DocumentationTok{\#\# v ggplot2   3.5.0     v tibble    3.2.1}
\DocumentationTok{\#\# v lubridate 1.9.3     v tidyr     1.3.1}
\DocumentationTok{\#\# v purrr     1.0.2     }
\DocumentationTok{\#\# {-}{-} Conflicts {-}{-}{-}{-}{-}{-}{-}{-}{-}{-}{-}{-}{-}{-}{-}{-}{-}{-}{-}{-}{-}{-}{-}{-}{-}{-}{-}{-}{-}{-}{-}{-}{-}{-}{-}{-}{-}{-}{-}{-}{-}{-} tidyverse\_conflicts() {-}{-}}
\DocumentationTok{\#\# x dplyr::filter() masks stats::filter()}
\DocumentationTok{\#\# x dplyr::lag()    masks stats::lag()}
\DocumentationTok{\#\# i Use the conflicted package (\textless{}http://conflicted.r{-}lib.org/\textgreater{}) to force all conflicts to become errors}
\end{Highlighting}
\end{Shaded}

\hypertarget{ux7df4ux7fd2ux554fux984c-2-1-ux78baux8a8d}{%
\section*{練習問題 2-1 {[}確認{]}}\label{ux7df4ux7fd2ux554fux984c-2-1-ux78baux8a8d}}
\addcontentsline{toc}{section}{練習問題 2-1 {[}確認{]}}

\hypertarget{ux7df4ux7fd2ux554fux984c-2-2-ux78baux8a8d}{%
\section*{練習問題 2-2 {[}確認{]}}\label{ux7df4ux7fd2ux554fux984c-2-2-ux78baux8a8d}}
\addcontentsline{toc}{section}{練習問題 2-2 {[}確認{]}}

\hypertarget{ux7df4ux7fd2ux554fux984c-2-3-ux78baux8a8d}{%
\section*{練習問題 2-3 {[}確認{]}}\label{ux7df4ux7fd2ux554fux984c-2-3-ux78baux8a8d}}
\addcontentsline{toc}{section}{練習問題 2-3 {[}確認{]}}

\hypertarget{ux7df4ux7fd2ux554fux984c-2-4-ux78baux8a8d}{%
\section*{練習問題 2-4 {[}確認{]}}\label{ux7df4ux7fd2ux554fux984c-2-4-ux78baux8a8d}}
\addcontentsline{toc}{section}{練習問題 2-4 {[}確認{]}}

\hypertarget{ux7df4ux7fd2ux554fux984c-2-5-ux78baux8a8d}{%
\section*{練習問題 2-5 {[}確認{]}}\label{ux7df4ux7fd2ux554fux984c-2-5-ux78baux8a8d}}
\addcontentsline{toc}{section}{練習問題 2-5 {[}確認{]}}

\hypertarget{ux7df4ux7fd2ux554fux984c-2-6-ux78baux8a8d}{%
\section*{練習問題 2-6 {[}確認{]}}\label{ux7df4ux7fd2ux554fux984c-2-6-ux78baux8a8d}}
\addcontentsline{toc}{section}{練習問題 2-6 {[}確認{]}}

\hypertarget{ux7df4ux7fd2ux554fux984c-2-7-ux767aux5c55}{%
\section*{練習問題 2-7 {[}発展{]}}\label{ux7df4ux7fd2ux554fux984c-2-7-ux767aux5c55}}
\addcontentsline{toc}{section}{練習問題 2-7 {[}発展{]}}

\hypertarget{ux7df4ux7fd2ux554fux984c-2-8-ux767aux5c55}{%
\section*{練習問題 2-8 {[}発展{]}}\label{ux7df4ux7fd2ux554fux984c-2-8-ux767aux5c55}}
\addcontentsline{toc}{section}{練習問題 2-8 {[}発展{]}}

\hypertarget{ux7df4ux7fd2ux554fux984c-2-9-ux767aux5c55}{%
\section*{練習問題 2-9 {[}発展{]}}\label{ux7df4ux7fd2ux554fux984c-2-9-ux767aux5c55}}
\addcontentsline{toc}{section}{練習問題 2-9 {[}発展{]}}

\hypertarget{ux7df4ux7fd2ux554fux984c-2-10-ux767aux5c55}{%
\section*{練習問題 2-10 {[}発展{]}}\label{ux7df4ux7fd2ux554fux984c-2-10-ux767aux5c55}}
\addcontentsline{toc}{section}{練習問題 2-10 {[}発展{]}}

\hypertarget{ux7df4ux7fd2ux554fux984c-2-11-ux78baux8a8d}{%
\section*{練習問題 2-11 {[}確認{]}}\label{ux7df4ux7fd2ux554fux984c-2-11-ux78baux8a8d}}
\addcontentsline{toc}{section}{練習問題 2-11 {[}確認{]}}

\hypertarget{ux7df4ux7fd2ux554fux984c-2-12-ux767aux5c55}{%
\section*{練習問題 2-12 {[}発展{]}}\label{ux7df4ux7fd2ux554fux984c-2-12-ux767aux5c55}}
\addcontentsline{toc}{section}{練習問題 2-12 {[}発展{]}}

\hypertarget{ux7df4ux7fd2ux554fux984c-2-13-ux78baux8a8d}{%
\section*{練習問題 2-13 {[}確認{]}}\label{ux7df4ux7fd2ux554fux984c-2-13-ux78baux8a8d}}
\addcontentsline{toc}{section}{練習問題 2-13 {[}確認{]}}

\hypertarget{ux7df4ux7fd2ux554fux984c-2-14-ux78baux8a8d}{%
\section*{練習問題 2-14 {[}確認{]}}\label{ux7df4ux7fd2ux554fux984c-2-14-ux78baux8a8d}}
\addcontentsline{toc}{section}{練習問題 2-14 {[}確認{]}}

\hypertarget{ch3}{%
\chapter*{第3章 統計理論の基礎}\label{ch3}}
\addcontentsline{toc}{chapter}{第3章 統計理論の基礎}

第3章では第2章と同じデータを使うため、新たなダウンロードは不要.
なお\href{https://www.yuhikaku.co.jp/books/detail/9784641053854}{出版社サイト}にある第3章のファイルは, 誤ったデータが格納されている.

必要なライブラリを読み込む.

\begin{Shaded}
\begin{Highlighting}[]
\FunctionTok{library}\NormalTok{(tidyverse)}
\end{Highlighting}
\end{Shaded}

\hypertarget{ux7df4ux7fd2ux554fux984c-3-1-ux78baux8a8d}{%
\section*{練習問題 3-1 {[}確認{]}}\label{ux7df4ux7fd2ux554fux984c-3-1-ux78baux8a8d}}
\addcontentsline{toc}{section}{練習問題 3-1 {[}確認{]}}

\hypertarget{ux7df4ux7fd2ux554fux984c-3-2-ux78baux8a8d}{%
\section*{練習問題 3-2 {[}確認{]}}\label{ux7df4ux7fd2ux554fux984c-3-2-ux78baux8a8d}}
\addcontentsline{toc}{section}{練習問題 3-2 {[}確認{]}}

\hypertarget{ux7df4ux7fd2ux554fux984c-3-3-ux78baux8a8d}{%
\section*{練習問題 3-3 {[}確認{]}}\label{ux7df4ux7fd2ux554fux984c-3-3-ux78baux8a8d}}
\addcontentsline{toc}{section}{練習問題 3-3 {[}確認{]}}

\hypertarget{ux7df4ux7fd2ux554fux984c-3-4-ux767aux5c55}{%
\section*{練習問題 3-4 {[}発展{]}}\label{ux7df4ux7fd2ux554fux984c-3-4-ux767aux5c55}}
\addcontentsline{toc}{section}{練習問題 3-4 {[}発展{]}}

\hypertarget{ch4}{%
\chapter*{第4章 線形単回帰モデルの推定と検定}\label{ch4}}
\addcontentsline{toc}{chapter}{第4章 線形単回帰モデルの推定と検定}

先に\href{https://www.yuhikaku.co.jp/books/detail/9784641053854}{出版社サイト}よりデータをダウンロードする.

\begin{Shaded}
\begin{Highlighting}[]
\CommentTok{\# サポートファイルへのリンク}
\NormalTok{curl }\OtherTok{\textless{}{-}} \StringTok{"https://www.yuhikaku.co.jp/static\_files/05385\_support04.zip"}
\CommentTok{\# ダウンロード保存用フォルダが存在しない場合, 作成}
\ControlFlowTok{if}\NormalTok{(}\SpecialCharTok{!}\FunctionTok{dir.exists}\NormalTok{(}\StringTok{"downloads"}\NormalTok{))\{}
    \FunctionTok{dir.create}\NormalTok{(}\StringTok{"downloads"}\NormalTok{)}
\NormalTok{\}}
\NormalTok{cdestfile }\OtherTok{\textless{}{-}} \StringTok{"downloads/support04.zip"}
\FunctionTok{download.file}\NormalTok{(curl, cdestfile)}
\CommentTok{\# データ保存用フォルダが存在しない場合, 作成}
\ControlFlowTok{if}\NormalTok{(}\SpecialCharTok{!}\FunctionTok{dir.exists}\NormalTok{(}\StringTok{"data"}\NormalTok{))\{}
    \FunctionTok{dir.create}\NormalTok{(}\StringTok{"data"}\NormalTok{)}
\NormalTok{\}}
\CommentTok{\# WSL上のRで解凍すると文字化けするので、Linuxのコマンドを外部呼び出し}
\CommentTok{\# Windowsの場合は別途コマンドを用いる.}
\ControlFlowTok{if}\NormalTok{(.Platform}\SpecialCharTok{$}\NormalTok{OS.type }\SpecialCharTok{==} \StringTok{"unix"}\NormalTok{) \{}
    \FunctionTok{system}\NormalTok{(}\FunctionTok{sprintf}\NormalTok{(}\StringTok{\textquotesingle{}unzip {-}n {-}Ocp932 \%s {-}d \%s\textquotesingle{}}\NormalTok{, }\StringTok{"downloads/support04.zip"}\NormalTok{, }\StringTok{"./data"}\NormalTok{))}
\NormalTok{\} }\ControlFlowTok{else}\NormalTok{ \{}
    \FunctionTok{print}\NormalTok{(}\StringTok{"Windowsで解凍するコマンドを別途追加せよ."}\NormalTok{)}
\NormalTok{\}}
\end{Highlighting}
\end{Shaded}

必要なライブラリを読み込む.

\begin{Shaded}
\begin{Highlighting}[]
\FunctionTok{library}\NormalTok{(tidyverse)}
\FunctionTok{library}\NormalTok{(openxlsx) }\CommentTok{\# Excelファイルを読み取るためのパッケージ}
\end{Highlighting}
\end{Shaded}

\hypertarget{ux7df4ux7fd2ux554fux984c-4-1-ux78baux8a8d}{%
\section*{練習問題 4-1 {[}確認{]}}\label{ux7df4ux7fd2ux554fux984c-4-1-ux78baux8a8d}}
\addcontentsline{toc}{section}{練習問題 4-1 {[}確認{]}}

\hypertarget{ux7df4ux7fd2ux554fux984c-4-2-ux5b9fux8a3c}{%
\section*{練習問題 4-2 {[}実証{]}}\label{ux7df4ux7fd2ux554fux984c-4-2-ux5b9fux8a3c}}
\addcontentsline{toc}{section}{練習問題 4-2 {[}実証{]}}

\hypertarget{ux7df4ux7fd2ux554fux984c-4-3-ux78baux8a8d}{%
\section*{練習問題 4-3 {[}確認{]}}\label{ux7df4ux7fd2ux554fux984c-4-3-ux78baux8a8d}}
\addcontentsline{toc}{section}{練習問題 4-3 {[}確認{]}}

\hypertarget{ux7df4ux7fd2ux554fux984c-4-4-ux78baux8a8d}{%
\section*{練習問題 4-4 {[}確認{]}}\label{ux7df4ux7fd2ux554fux984c-4-4-ux78baux8a8d}}
\addcontentsline{toc}{section}{練習問題 4-4 {[}確認{]}}

\hypertarget{ux7df4ux7fd2ux554fux984c-4-5-ux78baux8a8d}{%
\section*{練習問題 4-5 {[}確認{]}}\label{ux7df4ux7fd2ux554fux984c-4-5-ux78baux8a8d}}
\addcontentsline{toc}{section}{練習問題 4-5 {[}確認{]}}

\hypertarget{ux7df4ux7fd2ux554fux984c-4-6-ux78baux8a8d}{%
\section*{練習問題 4-6 {[}確認{]}}\label{ux7df4ux7fd2ux554fux984c-4-6-ux78baux8a8d}}
\addcontentsline{toc}{section}{練習問題 4-6 {[}確認{]}}

\hypertarget{ux7df4ux7fd2ux554fux984c-4-7-ux78baux8a8d}{%
\section*{練習問題 4-7 {[}確認{]}}\label{ux7df4ux7fd2ux554fux984c-4-7-ux78baux8a8d}}
\addcontentsline{toc}{section}{練習問題 4-7 {[}確認{]}}

\hypertarget{ux7df4ux7fd2ux554fux984c-4-8-ux767aux5c55}{%
\section*{練習問題 4-8 {[}発展{]}}\label{ux7df4ux7fd2ux554fux984c-4-8-ux767aux5c55}}
\addcontentsline{toc}{section}{練習問題 4-8 {[}発展{]}}

\hypertarget{ux7df4ux7fd2ux554fux984c-4-9-ux767aux5c55}{%
\section*{練習問題 4-9 {[}発展{]}}\label{ux7df4ux7fd2ux554fux984c-4-9-ux767aux5c55}}
\addcontentsline{toc}{section}{練習問題 4-9 {[}発展{]}}

\hypertarget{ux7df4ux7fd2ux554fux984c-4-10-ux5b9fux8a3c}{%
\section*{練習問題 4-10 {[}実証{]}}\label{ux7df4ux7fd2ux554fux984c-4-10-ux5b9fux8a3c}}
\addcontentsline{toc}{section}{練習問題 4-10 {[}実証{]}}

\hypertarget{ch5}{%
\chapter*{第5章 重回帰モデルの推定と検定}\label{ch5}}
\addcontentsline{toc}{chapter}{第5章 重回帰モデルの推定と検定}

先に\href{https://www.yuhikaku.co.jp/books/detail/9784641053854}{出版社サイト}よりデータをダウンロードする.

\begin{Shaded}
\begin{Highlighting}[]
\CommentTok{\# サポートファイルへのリンク}
\NormalTok{curl }\OtherTok{\textless{}{-}} \StringTok{"https://www.yuhikaku.co.jp/static\_files/05385\_support05.zip"}
\CommentTok{\# ダウンロード保存用フォルダが存在しない場合, 作成}
\ControlFlowTok{if}\NormalTok{(}\SpecialCharTok{!}\FunctionTok{dir.exists}\NormalTok{(}\StringTok{"downloads"}\NormalTok{))\{}
    \FunctionTok{dir.create}\NormalTok{(}\StringTok{"downloads"}\NormalTok{)}
\NormalTok{\}}
\NormalTok{cdestfile }\OtherTok{\textless{}{-}} \StringTok{"downloads/support05.zip"}
\FunctionTok{download.file}\NormalTok{(curl, cdestfile)}
\CommentTok{\# データ保存用フォルダが存在しない場合, 作成}
\ControlFlowTok{if}\NormalTok{(}\SpecialCharTok{!}\FunctionTok{dir.exists}\NormalTok{(}\StringTok{"data"}\NormalTok{))\{}
    \FunctionTok{dir.create}\NormalTok{(}\StringTok{"data"}\NormalTok{)}
\NormalTok{\}}
\CommentTok{\# WSL上のRで解凍すると文字化けするので、Linuxのコマンドを外部呼び出し}
\CommentTok{\# Windowsの場合は別途コマンドを用いる.}
\ControlFlowTok{if}\NormalTok{(.Platform}\SpecialCharTok{$}\NormalTok{OS.type }\SpecialCharTok{==} \StringTok{"unix"}\NormalTok{) \{}
    \FunctionTok{system}\NormalTok{(}\FunctionTok{sprintf}\NormalTok{(}\StringTok{\textquotesingle{}unzip {-}n {-}Ocp932 \%s {-}d \%s\textquotesingle{}}\NormalTok{, }\StringTok{"downloads/support05.zip"}\NormalTok{, }\StringTok{"./data"}\NormalTok{))}
\NormalTok{\} }\ControlFlowTok{else}\NormalTok{ \{}
    \FunctionTok{print}\NormalTok{(}\StringTok{"Windowsで解凍するコマンドを別途追加せよ."}\NormalTok{)}
\NormalTok{\}}
\end{Highlighting}
\end{Shaded}

必要なライブラリを読み込む.

\begin{Shaded}
\begin{Highlighting}[]
\FunctionTok{library}\NormalTok{(tidyverse)}
\end{Highlighting}
\end{Shaded}

\hypertarget{ux7df4ux7fd2ux554fux984c-5-1-ux78baux8a8d}{%
\section*{練習問題 5-1 {[}確認{]}}\label{ux7df4ux7fd2ux554fux984c-5-1-ux78baux8a8d}}
\addcontentsline{toc}{section}{練習問題 5-1 {[}確認{]}}

\hypertarget{ux7df4ux7fd2ux554fux984c-5-2-ux78baux8a8d}{%
\section*{練習問題 5-2 {[}確認{]}}\label{ux7df4ux7fd2ux554fux984c-5-2-ux78baux8a8d}}
\addcontentsline{toc}{section}{練習問題 5-2 {[}確認{]}}

\hypertarget{ux7df4ux7fd2ux554fux984c-5-3-ux78baux8a8d}{%
\section*{練習問題 5-3 {[}確認{]}}\label{ux7df4ux7fd2ux554fux984c-5-3-ux78baux8a8d}}
\addcontentsline{toc}{section}{練習問題 5-3 {[}確認{]}}

\hypertarget{ux7df4ux7fd2ux554fux984c-5-4-ux78baux8a8d}{%
\section*{練習問題 5-4 {[}確認{]}}\label{ux7df4ux7fd2ux554fux984c-5-4-ux78baux8a8d}}
\addcontentsline{toc}{section}{練習問題 5-4 {[}確認{]}}

\hypertarget{ux7df4ux7fd2ux554fux984c-5-5-ux78baux8a8d}{%
\section*{練習問題 5-5 {[}確認{]}}\label{ux7df4ux7fd2ux554fux984c-5-5-ux78baux8a8d}}
\addcontentsline{toc}{section}{練習問題 5-5 {[}確認{]}}

\hypertarget{ux7df4ux7fd2ux554fux984c-5-6-ux78baux8a8d}{%
\section*{練習問題 5-6 {[}確認{]}}\label{ux7df4ux7fd2ux554fux984c-5-6-ux78baux8a8d}}
\addcontentsline{toc}{section}{練習問題 5-6 {[}確認{]}}

\hypertarget{ux7df4ux7fd2ux554fux984c-5-7-ux78baux8a8d}{%
\section*{練習問題 5-7 {[}確認{]}}\label{ux7df4ux7fd2ux554fux984c-5-7-ux78baux8a8d}}
\addcontentsline{toc}{section}{練習問題 5-7 {[}確認{]}}

\hypertarget{ux7df4ux7fd2ux554fux984c-5-8-ux78baux8a8d}{%
\section*{練習問題 5-8 {[}確認{]}}\label{ux7df4ux7fd2ux554fux984c-5-8-ux78baux8a8d}}
\addcontentsline{toc}{section}{練習問題 5-8 {[}確認{]}}

\hypertarget{ux7df4ux7fd2ux554fux984c-5-9-ux78baux8a8d}{%
\section*{練習問題 5-9 {[}確認{]}}\label{ux7df4ux7fd2ux554fux984c-5-9-ux78baux8a8d}}
\addcontentsline{toc}{section}{練習問題 5-9 {[}確認{]}}

\hypertarget{ux7df4ux7fd2ux554fux984c-5-10-ux767aux5c55}{%
\section*{練習問題 5-10 {[}発展{]}}\label{ux7df4ux7fd2ux554fux984c-5-10-ux767aux5c55}}
\addcontentsline{toc}{section}{練習問題 5-10 {[}発展{]}}

\hypertarget{ux7df4ux7fd2ux554fux984c-5-11-ux767aux5c55}{%
\section*{練習問題 5-11 {[}発展{]}}\label{ux7df4ux7fd2ux554fux984c-5-11-ux767aux5c55}}
\addcontentsline{toc}{section}{練習問題 5-11 {[}発展{]}}

\hypertarget{ux7df4ux7fd2ux554fux984c-5-12-ux767aux5c55}{%
\section*{練習問題 5-12 {[}発展{]}}\label{ux7df4ux7fd2ux554fux984c-5-12-ux767aux5c55}}
\addcontentsline{toc}{section}{練習問題 5-12 {[}発展{]}}

\hypertarget{ux7df4ux7fd2ux554fux984c-5-13-ux767aux5c55}{%
\section*{練習問題 5-13 {[}*発展{]}}\label{ux7df4ux7fd2ux554fux984c-5-13-ux767aux5c55}}
\addcontentsline{toc}{section}{練習問題 5-13 {[}*発展{]}}

\hypertarget{ux7df4ux7fd2ux554fux984c-5-14-ux5b9fux8a3c}{%
\section*{練習問題 5-14 {[}実証{]}}\label{ux7df4ux7fd2ux554fux984c-5-14-ux5b9fux8a3c}}
\addcontentsline{toc}{section}{練習問題 5-14 {[}実証{]}}

\hypertarget{ux7df4ux7fd2ux554fux984c-5-15-ux5b9fux8a3c}{%
\section*{練習問題 5-15 {[}実証{]}}\label{ux7df4ux7fd2ux554fux984c-5-15-ux5b9fux8a3c}}
\addcontentsline{toc}{section}{練習問題 5-15 {[}実証{]}}

\hypertarget{ch6}{%
\chapter*{第6章 パネルデータ分析}\label{ch6}}
\addcontentsline{toc}{chapter}{第6章 パネルデータ分析}

先に\href{https://www.yuhikaku.co.jp/books/detail/9784641053854}{出版社サイト}よりデータをダウンロードする.

\begin{Shaded}
\begin{Highlighting}[]
\CommentTok{\# サポートファイルへのリンク}
\NormalTok{curl }\OtherTok{\textless{}{-}} \StringTok{"https://www.yuhikaku.co.jp/static\_files/05385\_support06.zip"}
\CommentTok{\# ダウンロード保存用フォルダが存在しない場合, 作成}
\ControlFlowTok{if}\NormalTok{(}\SpecialCharTok{!}\FunctionTok{dir.exists}\NormalTok{(}\StringTok{"downloads"}\NormalTok{))\{}
    \FunctionTok{dir.create}\NormalTok{(}\StringTok{"downloads"}\NormalTok{)}
\NormalTok{\}}
\NormalTok{cdestfile }\OtherTok{\textless{}{-}} \StringTok{"downloads/support06.zip"}
\FunctionTok{download.file}\NormalTok{(curl, cdestfile)}
\CommentTok{\# データ保存用フォルダが存在しない場合, 作成}
\ControlFlowTok{if}\NormalTok{(}\SpecialCharTok{!}\FunctionTok{dir.exists}\NormalTok{(}\StringTok{"data"}\NormalTok{))\{}
    \FunctionTok{dir.create}\NormalTok{(}\StringTok{"data"}\NormalTok{)}
\NormalTok{\}}
\CommentTok{\# WSL上のRで解凍すると文字化けするので、Linuxのコマンドを外部呼び出し}
\CommentTok{\# Windowsの場合は別途コマンドを用いる.}
\ControlFlowTok{if}\NormalTok{(.Platform}\SpecialCharTok{$}\NormalTok{OS.type }\SpecialCharTok{==} \StringTok{"unix"}\NormalTok{) \{}
    \FunctionTok{system}\NormalTok{(}\FunctionTok{sprintf}\NormalTok{(}\StringTok{\textquotesingle{}unzip {-}n {-}Ocp932 \%s {-}d \%s\textquotesingle{}}\NormalTok{, }\StringTok{"downloads/support06.zip"}\NormalTok{, }\StringTok{"./data"}\NormalTok{))}
\NormalTok{\} }\ControlFlowTok{else}\NormalTok{ \{}
    \FunctionTok{print}\NormalTok{(}\StringTok{"Windowsで解凍するコマンドを別途追加せよ."}\NormalTok{)}
\NormalTok{\}}
\end{Highlighting}
\end{Shaded}

\hypertarget{ux7df4ux7fd2ux554fux984c-6-1-ux78baux8a8d}{%
\section*{練習問題 6-1 {[}確認{]}}\label{ux7df4ux7fd2ux554fux984c-6-1-ux78baux8a8d}}
\addcontentsline{toc}{section}{練習問題 6-1 {[}確認{]}}

\hypertarget{ux7df4ux7fd2ux554fux984c-6-2-ux78baux8a8d}{%
\section*{練習問題 6-2 {[}確認{]}}\label{ux7df4ux7fd2ux554fux984c-6-2-ux78baux8a8d}}
\addcontentsline{toc}{section}{練習問題 6-2 {[}確認{]}}

\hypertarget{ux7df4ux7fd2ux554fux984c-6-3-ux78baux8a8d}{%
\section*{練習問題 6-3 {[}確認{]}}\label{ux7df4ux7fd2ux554fux984c-6-3-ux78baux8a8d}}
\addcontentsline{toc}{section}{練習問題 6-3 {[}確認{]}}

\hypertarget{ux7df4ux7fd2ux554fux984c-6-4-ux78baux8a8d}{%
\section*{練習問題 6-4 {[}確認{]}}\label{ux7df4ux7fd2ux554fux984c-6-4-ux78baux8a8d}}
\addcontentsline{toc}{section}{練習問題 6-4 {[}確認{]}}

\hypertarget{ux7df4ux7fd2ux554fux984c-6-5-ux78baux8a8d}{%
\section*{練習問題 6-5 {[}確認{]}}\label{ux7df4ux7fd2ux554fux984c-6-5-ux78baux8a8d}}
\addcontentsline{toc}{section}{練習問題 6-5 {[}確認{]}}

\hypertarget{ux7df4ux7fd2ux554fux984c-6-6-ux78baux8a8d}{%
\section*{練習問題 6-6 {[}確認{]}}\label{ux7df4ux7fd2ux554fux984c-6-6-ux78baux8a8d}}
\addcontentsline{toc}{section}{練習問題 6-6 {[}確認{]}}

\hypertarget{ux7df4ux7fd2ux554fux984c-6-7-ux78baux8a8d}{%
\section*{練習問題 6-7 {[}確認{]}}\label{ux7df4ux7fd2ux554fux984c-6-7-ux78baux8a8d}}
\addcontentsline{toc}{section}{練習問題 6-7 {[}確認{]}}

\hypertarget{ux7df4ux7fd2ux554fux984c-6-8-ux767aux5c55}{%
\section*{練習問題 6-8 {[}発展{]}}\label{ux7df4ux7fd2ux554fux984c-6-8-ux767aux5c55}}
\addcontentsline{toc}{section}{練習問題 6-8 {[}発展{]}}

\hypertarget{ux7df4ux7fd2ux554fux984c-6-9-ux767aux5c55}{%
\section*{練習問題 6-9 {[}発展{]}}\label{ux7df4ux7fd2ux554fux984c-6-9-ux767aux5c55}}
\addcontentsline{toc}{section}{練習問題 6-9 {[}発展{]}}

\hypertarget{ux7df4ux7fd2ux554fux984c-6-10-ux5b9fux8a3c}{%
\section*{練習問題 6-10 {[}実証{]}}\label{ux7df4ux7fd2ux554fux984c-6-10-ux5b9fux8a3c}}
\addcontentsline{toc}{section}{練習問題 6-10 {[}実証{]}}

\hypertarget{ux7df4ux7fd2ux554fux984c-6-11-ux5b9fux8a3c}{%
\section*{練習問題 6-11 {[}実証{]}}\label{ux7df4ux7fd2ux554fux984c-6-11-ux5b9fux8a3c}}
\addcontentsline{toc}{section}{練習問題 6-11 {[}実証{]}}

\hypertarget{ch7}{%
\chapter*{第7章 操作変数法}\label{ch7}}
\addcontentsline{toc}{chapter}{第7章 操作変数法}

先に\href{https://www.yuhikaku.co.jp/books/detail/9784641053854}{出版社サイト}よりデータをダウンロードする.

\begin{Shaded}
\begin{Highlighting}[]
\CommentTok{\# サポートファイルへのリンク}
\NormalTok{curl }\OtherTok{\textless{}{-}} \StringTok{"https://www.yuhikaku.co.jp/static\_files/05385\_support07.zip"}
\CommentTok{\# ダウンロード保存用フォルダが存在しない場合, 作成}
\ControlFlowTok{if}\NormalTok{(}\SpecialCharTok{!}\FunctionTok{dir.exists}\NormalTok{(}\StringTok{"downloads"}\NormalTok{))\{}
    \FunctionTok{dir.create}\NormalTok{(}\StringTok{"downloads"}\NormalTok{)}
\NormalTok{\}}
\NormalTok{cdestfile }\OtherTok{\textless{}{-}} \StringTok{"downloads/support07.zip"}
\FunctionTok{download.file}\NormalTok{(curl, cdestfile)}
\CommentTok{\# データ保存用フォルダが存在しない場合, 作成}
\ControlFlowTok{if}\NormalTok{(}\SpecialCharTok{!}\FunctionTok{dir.exists}\NormalTok{(}\StringTok{"data"}\NormalTok{))\{}
    \FunctionTok{dir.create}\NormalTok{(}\StringTok{"data"}\NormalTok{)}
\NormalTok{\}}
\CommentTok{\# WSL上のRで解凍すると文字化けするので、Linuxのコマンドを外部呼び出し}
\CommentTok{\# Windowsの場合は別途コマンドを用いる.}
\ControlFlowTok{if}\NormalTok{(.Platform}\SpecialCharTok{$}\NormalTok{OS.type }\SpecialCharTok{==} \StringTok{"unix"}\NormalTok{) \{}
    \FunctionTok{system}\NormalTok{(}\FunctionTok{sprintf}\NormalTok{(}\StringTok{\textquotesingle{}unzip {-}n {-}Ocp932 \%s {-}d \%s\textquotesingle{}}\NormalTok{, }\StringTok{"downloads/support07.zip"}\NormalTok{, }\StringTok{"./data"}\NormalTok{))}
\NormalTok{\} }\ControlFlowTok{else}\NormalTok{ \{}
    \FunctionTok{print}\NormalTok{(}\StringTok{"Windowsで解凍するコマンドを別途追加せよ."}\NormalTok{)}
\NormalTok{\}}
\end{Highlighting}
\end{Shaded}

\hypertarget{ux7df4ux7fd2ux554fux984c-7-1-ux78baux8a8d}{%
\section*{練習問題 7-1 {[}確認{]}}\label{ux7df4ux7fd2ux554fux984c-7-1-ux78baux8a8d}}
\addcontentsline{toc}{section}{練習問題 7-1 {[}確認{]}}

\hypertarget{ux7df4ux7fd2ux554fux984c-7-2-ux78baux8a8d}{%
\section*{練習問題 7-2 {[}確認{]}}\label{ux7df4ux7fd2ux554fux984c-7-2-ux78baux8a8d}}
\addcontentsline{toc}{section}{練習問題 7-2 {[}確認{]}}

\hypertarget{ux7df4ux7fd2ux554fux984c-7-3-ux78baux8a8d}{%
\section*{練習問題 7-3 {[}確認{]}}\label{ux7df4ux7fd2ux554fux984c-7-3-ux78baux8a8d}}
\addcontentsline{toc}{section}{練習問題 7-3 {[}確認{]}}

\hypertarget{ux7df4ux7fd2ux554fux984c-7-4-ux78baux8a8d}{%
\section*{練習問題 7-4 {[}確認{]}}\label{ux7df4ux7fd2ux554fux984c-7-4-ux78baux8a8d}}
\addcontentsline{toc}{section}{練習問題 7-4 {[}確認{]}}

\hypertarget{ux7df4ux7fd2ux554fux984c-7-5-ux78baux8a8d}{%
\section*{練習問題 7-5 {[}確認{]}}\label{ux7df4ux7fd2ux554fux984c-7-5-ux78baux8a8d}}
\addcontentsline{toc}{section}{練習問題 7-5 {[}確認{]}}

\hypertarget{ux7df4ux7fd2ux554fux984c-7-6-ux78baux8a8d}{%
\section*{練習問題 7-6 {[}確認{]}}\label{ux7df4ux7fd2ux554fux984c-7-6-ux78baux8a8d}}
\addcontentsline{toc}{section}{練習問題 7-6 {[}確認{]}}

\hypertarget{ux7df4ux7fd2ux554fux984c-7-7-ux78baux8a8d}{%
\section*{練習問題 7-7 {[}確認{]}}\label{ux7df4ux7fd2ux554fux984c-7-7-ux78baux8a8d}}
\addcontentsline{toc}{section}{練習問題 7-7 {[}確認{]}}

\hypertarget{ux7df4ux7fd2ux554fux984c-7-8-ux78baux8a8d}{%
\section*{練習問題 7-8 {[}確認{]}}\label{ux7df4ux7fd2ux554fux984c-7-8-ux78baux8a8d}}
\addcontentsline{toc}{section}{練習問題 7-8 {[}確認{]}}

\hypertarget{ux7df4ux7fd2ux554fux984c-7-9-ux78baux8a8d}{%
\section*{練習問題 7-9 {[}確認{]}}\label{ux7df4ux7fd2ux554fux984c-7-9-ux78baux8a8d}}
\addcontentsline{toc}{section}{練習問題 7-9 {[}確認{]}}

\hypertarget{ux7df4ux7fd2ux554fux984c-7-10-ux78baux8a8d}{%
\section*{練習問題 7-10 {[}確認{]}}\label{ux7df4ux7fd2ux554fux984c-7-10-ux78baux8a8d}}
\addcontentsline{toc}{section}{練習問題 7-10 {[}確認{]}}

\hypertarget{ux7df4ux7fd2ux554fux984c-7-11-ux767aux5c55}{%
\section*{練習問題 7-11 {[}発展{]}}\label{ux7df4ux7fd2ux554fux984c-7-11-ux767aux5c55}}
\addcontentsline{toc}{section}{練習問題 7-11 {[}発展{]}}

\hypertarget{ux7df4ux7fd2ux554fux984c-7-12-ux767aux5c55}{%
\section*{練習問題 7-12 {[}発展{]}}\label{ux7df4ux7fd2ux554fux984c-7-12-ux767aux5c55}}
\addcontentsline{toc}{section}{練習問題 7-12 {[}発展{]}}

\hypertarget{ux7df4ux7fd2ux554fux984c-7-13-ux5b9fux8a3c}{%
\section*{練習問題 7-13 {[}実証{]}}\label{ux7df4ux7fd2ux554fux984c-7-13-ux5b9fux8a3c}}
\addcontentsline{toc}{section}{練習問題 7-13 {[}実証{]}}

\hypertarget{ch8}{%
\chapter*{第8章 制限従属変数モデル}\label{ch8}}
\addcontentsline{toc}{chapter}{第8章 制限従属変数モデル}

先に\href{https://www.yuhikaku.co.jp/books/detail/9784641053854}{出版社サイト}よりデータをダウンロードする.

\begin{Shaded}
\begin{Highlighting}[]
\CommentTok{\# サポートファイルへのリンク}
\NormalTok{curl }\OtherTok{\textless{}{-}} \StringTok{"https://www.yuhikaku.co.jp/static\_files/05385\_support08.zip"}
\CommentTok{\# ダウンロード保存用フォルダが存在しない場合, 作成}
\ControlFlowTok{if}\NormalTok{(}\SpecialCharTok{!}\FunctionTok{dir.exists}\NormalTok{(}\StringTok{"downloads"}\NormalTok{))\{}
    \FunctionTok{dir.create}\NormalTok{(}\StringTok{"downloads"}\NormalTok{)}
\NormalTok{\}}
\NormalTok{cdestfile }\OtherTok{\textless{}{-}} \StringTok{"downloads/support08.zip"}
\FunctionTok{download.file}\NormalTok{(curl, cdestfile)}
\CommentTok{\# データ保存用フォルダが存在しない場合, 作成}
\ControlFlowTok{if}\NormalTok{(}\SpecialCharTok{!}\FunctionTok{dir.exists}\NormalTok{(}\StringTok{"data"}\NormalTok{))\{}
    \FunctionTok{dir.create}\NormalTok{(}\StringTok{"data"}\NormalTok{)}
\NormalTok{\}}
\CommentTok{\# WSL上のRで解凍すると文字化けするので、Linuxのコマンドを外部呼び出し}
\CommentTok{\# Windowsの場合は別途コマンドを用いる.}
\ControlFlowTok{if}\NormalTok{(.Platform}\SpecialCharTok{$}\NormalTok{OS.type }\SpecialCharTok{==} \StringTok{"unix"}\NormalTok{) \{}
    \FunctionTok{system}\NormalTok{(}\FunctionTok{sprintf}\NormalTok{(}\StringTok{\textquotesingle{}unzip {-}n {-}Ocp932 \%s {-}d \%s\textquotesingle{}}\NormalTok{, }\StringTok{"downloads/support08.zip"}\NormalTok{, }\StringTok{"./data"}\NormalTok{))}
\NormalTok{\} }\ControlFlowTok{else}\NormalTok{ \{}
    \FunctionTok{print}\NormalTok{(}\StringTok{"Windowsで解凍するコマンドを別途追加せよ."}\NormalTok{)}
\NormalTok{\}}
\end{Highlighting}
\end{Shaded}

\hypertarget{ux7df4ux7fd2ux554fux984c-8-1-ux78baux8a8d}{%
\section*{練習問題 8-1 {[}確認{]}}\label{ux7df4ux7fd2ux554fux984c-8-1-ux78baux8a8d}}
\addcontentsline{toc}{section}{練習問題 8-1 {[}確認{]}}

\hypertarget{ux7df4ux7fd2ux554fux984c-8-2-ux78baux8a8d}{%
\section*{練習問題 8-2 {[}確認{]}}\label{ux7df4ux7fd2ux554fux984c-8-2-ux78baux8a8d}}
\addcontentsline{toc}{section}{練習問題 8-2 {[}確認{]}}

\hypertarget{ux7df4ux7fd2ux554fux984c-8-3-ux767aux5c55}{%
\section*{練習問題 8-3 {[}発展{]}}\label{ux7df4ux7fd2ux554fux984c-8-3-ux767aux5c55}}
\addcontentsline{toc}{section}{練習問題 8-3 {[}発展{]}}

\hypertarget{ux7df4ux7fd2ux554fux984c-8-4-ux5b9fux8a3c}{%
\section*{練習問題 8-4 {[}実証{]}}\label{ux7df4ux7fd2ux554fux984c-8-4-ux5b9fux8a3c}}
\addcontentsline{toc}{section}{練習問題 8-4 {[}実証{]}}

\hypertarget{ch9}{%
\chapter*{第9章 政策評価モデル}\label{ch9}}
\addcontentsline{toc}{chapter}{第9章 政策評価モデル}

先に\href{https://www.yuhikaku.co.jp/books/detail/9784641053854}{出版社サイト}よりデータをダウンロードする.

\begin{Shaded}
\begin{Highlighting}[]
\CommentTok{\# サポートファイルへのリンク}
\NormalTok{curl }\OtherTok{\textless{}{-}} \StringTok{"https://www.yuhikaku.co.jp/static\_files/05385\_support09.zip"}
\CommentTok{\# ダウンロード保存用フォルダが存在しない場合, 作成}
\ControlFlowTok{if}\NormalTok{(}\SpecialCharTok{!}\FunctionTok{dir.exists}\NormalTok{(}\StringTok{"downloads"}\NormalTok{))\{}
    \FunctionTok{dir.create}\NormalTok{(}\StringTok{"downloads"}\NormalTok{)}
\NormalTok{\}}
\NormalTok{cdestfile }\OtherTok{\textless{}{-}} \StringTok{"downloads/support09.zip"}
\FunctionTok{download.file}\NormalTok{(curl, cdestfile)}
\CommentTok{\# データ保存用フォルダが存在しない場合, 作成}
\ControlFlowTok{if}\NormalTok{(}\SpecialCharTok{!}\FunctionTok{dir.exists}\NormalTok{(}\StringTok{"data"}\NormalTok{))\{}
    \FunctionTok{dir.create}\NormalTok{(}\StringTok{"data"}\NormalTok{)}
\NormalTok{\}}
\CommentTok{\# WSL上のRで解凍すると文字化けするので、Linuxのコマンドを外部呼び出し}
\CommentTok{\# Windowsの場合は別途コマンドを用いる.}
\ControlFlowTok{if}\NormalTok{(.Platform}\SpecialCharTok{$}\NormalTok{OS.type }\SpecialCharTok{==} \StringTok{"unix"}\NormalTok{) \{}
    \FunctionTok{system}\NormalTok{(}\FunctionTok{sprintf}\NormalTok{(}\StringTok{\textquotesingle{}unzip {-}n {-}Ocp932 \%s {-}d \%s\textquotesingle{}}\NormalTok{, }\StringTok{"downloads/support09.zip"}\NormalTok{, }\StringTok{"./data"}\NormalTok{))}
\NormalTok{\} }\ControlFlowTok{else}\NormalTok{ \{}
    \FunctionTok{print}\NormalTok{(}\StringTok{"Windowsで解凍するコマンドを別途追加せよ."}\NormalTok{)}
\NormalTok{\}}
\end{Highlighting}
\end{Shaded}

\hypertarget{ux7df4ux7fd2ux554fux984c-9-1-ux78baux8a8d}{%
\section*{練習問題 9-1 {[}確認{]}}\label{ux7df4ux7fd2ux554fux984c-9-1-ux78baux8a8d}}
\addcontentsline{toc}{section}{練習問題 9-1 {[}確認{]}}

\hypertarget{ux7df4ux7fd2ux554fux984c-9-2-ux78baux8a8d}{%
\section*{練習問題 9-2 {[}確認{]}}\label{ux7df4ux7fd2ux554fux984c-9-2-ux78baux8a8d}}
\addcontentsline{toc}{section}{練習問題 9-2 {[}確認{]}}

\hypertarget{ux7df4ux7fd2ux554fux984c-9-3-ux767aux5c55}{%
\section*{練習問題 9-3 {[}発展{]}}\label{ux7df4ux7fd2ux554fux984c-9-3-ux767aux5c55}}
\addcontentsline{toc}{section}{練習問題 9-3 {[}発展{]}}

\hypertarget{ux7df4ux7fd2ux554fux984c-9-4-ux767aux5c55}{%
\section*{練習問題 9-4 {[}発展{]}}\label{ux7df4ux7fd2ux554fux984c-9-4-ux767aux5c55}}
\addcontentsline{toc}{section}{練習問題 9-4 {[}発展{]}}

\hypertarget{ux7df4ux7fd2ux554fux984c-9-5-ux5b9fux8a3c}{%
\section*{練習問題 9-5 {[}実証{]}}\label{ux7df4ux7fd2ux554fux984c-9-5-ux5b9fux8a3c}}
\addcontentsline{toc}{section}{練習問題 9-5 {[}実証{]}}

\hypertarget{ch10}{%
\chapter*{第10章 系列相関と時系列モデル}\label{ch10}}
\addcontentsline{toc}{chapter}{第10章 系列相関と時系列モデル}

先に\href{https://www.yuhikaku.co.jp/books/detail/9784641053854}{出版社サイト}よりデータをダウンロードする.

\begin{Shaded}
\begin{Highlighting}[]
\CommentTok{\# サポートファイルへのリンク}
\NormalTok{curl }\OtherTok{\textless{}{-}} \StringTok{"https://www.yuhikaku.co.jp/static\_files/05385\_support10.zip"}
\CommentTok{\# ダウンロード保存用フォルダが存在しない場合, 作成}
\ControlFlowTok{if}\NormalTok{(}\SpecialCharTok{!}\FunctionTok{dir.exists}\NormalTok{(}\StringTok{"downloads"}\NormalTok{))\{}
    \FunctionTok{dir.create}\NormalTok{(}\StringTok{"downloads"}\NormalTok{)}
\NormalTok{\}}
\NormalTok{cdestfile }\OtherTok{\textless{}{-}} \StringTok{"downloads/support10.zip"}
\FunctionTok{download.file}\NormalTok{(curl, cdestfile)}
\CommentTok{\# データ保存用フォルダが存在しない場合, 作成}
\ControlFlowTok{if}\NormalTok{(}\SpecialCharTok{!}\FunctionTok{dir.exists}\NormalTok{(}\StringTok{"data"}\NormalTok{))\{}
    \FunctionTok{dir.create}\NormalTok{(}\StringTok{"data"}\NormalTok{)}
\NormalTok{\}}
\CommentTok{\# WSL上のRで解凍すると文字化けするので、Linuxのコマンドを外部呼び出し}
\CommentTok{\# Windowsの場合は別途コマンドを用いる.}
\ControlFlowTok{if}\NormalTok{(.Platform}\SpecialCharTok{$}\NormalTok{OS.type }\SpecialCharTok{==} \StringTok{"unix"}\NormalTok{) \{}
    \FunctionTok{system}\NormalTok{(}\FunctionTok{sprintf}\NormalTok{(}\StringTok{\textquotesingle{}unzip {-}n {-}Ocp932 \%s {-}d \%s\textquotesingle{}}\NormalTok{, }\StringTok{"downloads/support10.zip"}\NormalTok{, }\StringTok{"./data"}\NormalTok{))}
\NormalTok{\} }\ControlFlowTok{else}\NormalTok{ \{}
    \FunctionTok{print}\NormalTok{(}\StringTok{"Windowsで解凍するコマンドを別途追加せよ."}\NormalTok{)}
\NormalTok{\}}
\end{Highlighting}
\end{Shaded}

\hypertarget{ux7df4ux7fd2ux554fux984c-10-1-ux78baux8a8d}{%
\section*{練習問題 10-1 {[}確認{]}}\label{ux7df4ux7fd2ux554fux984c-10-1-ux78baux8a8d}}
\addcontentsline{toc}{section}{練習問題 10-1 {[}確認{]}}

\hypertarget{ux7df4ux7fd2ux554fux984c-10-2-ux78baux8a8d}{%
\section*{練習問題 10-2 {[}確認{]}}\label{ux7df4ux7fd2ux554fux984c-10-2-ux78baux8a8d}}
\addcontentsline{toc}{section}{練習問題 10-2 {[}確認{]}}

\hypertarget{ux7df4ux7fd2ux554fux984c-10-3-ux78baux8a8d}{%
\section*{練習問題 10-3 {[}確認{]}}\label{ux7df4ux7fd2ux554fux984c-10-3-ux78baux8a8d}}
\addcontentsline{toc}{section}{練習問題 10-3 {[}確認{]}}

\hypertarget{ux7df4ux7fd2ux554fux984c-10-4-ux78baux8a8d}{%
\section*{練習問題 10-4 {[}確認{]}}\label{ux7df4ux7fd2ux554fux984c-10-4-ux78baux8a8d}}
\addcontentsline{toc}{section}{練習問題 10-4 {[}確認{]}}

\hypertarget{ux7df4ux7fd2ux554fux984c-10-5-ux78baux8a8d}{%
\section*{練習問題 10-5 {[}確認{]}}\label{ux7df4ux7fd2ux554fux984c-10-5-ux78baux8a8d}}
\addcontentsline{toc}{section}{練習問題 10-5 {[}確認{]}}

\hypertarget{ux7df4ux7fd2ux554fux984c-10-6-ux78baux8a8d}{%
\section*{練習問題 10-6 {[}確認{]}}\label{ux7df4ux7fd2ux554fux984c-10-6-ux78baux8a8d}}
\addcontentsline{toc}{section}{練習問題 10-6 {[}確認{]}}

\hypertarget{ux7df4ux7fd2ux554fux984c-10-7-ux767aux5c55}{%
\section*{練習問題 10-7 {[}発展{]}}\label{ux7df4ux7fd2ux554fux984c-10-7-ux767aux5c55}}
\addcontentsline{toc}{section}{練習問題 10-7 {[}発展{]}}

\hypertarget{ux7df4ux7fd2ux554fux984c-10-8-ux767aux5c55}{%
\section*{練習問題 10-8 {[}発展{]}}\label{ux7df4ux7fd2ux554fux984c-10-8-ux767aux5c55}}
\addcontentsline{toc}{section}{練習問題 10-8 {[}発展{]}}

\hypertarget{ux7df4ux7fd2ux554fux984c-10-9-ux5b9fux8a3c}{%
\section*{練習問題 10-9 {[}実証{]}}\label{ux7df4ux7fd2ux554fux984c-10-9-ux5b9fux8a3c}}
\addcontentsline{toc}{section}{練習問題 10-9 {[}実証{]}}

\hypertarget{ux7df4ux7fd2ux554fux984c-10-10-ux5b9fux8a3c}{%
\section*{練習問題 10-10 {[}実証{]}}\label{ux7df4ux7fd2ux554fux984c-10-10-ux5b9fux8a3c}}
\addcontentsline{toc}{section}{練習問題 10-10 {[}実証{]}}

\hypertarget{ch11}{%
\chapter*{第11章 トレンドと構造変化}\label{ch11}}
\addcontentsline{toc}{chapter}{第11章 トレンドと構造変化}

先に\href{https://www.yuhikaku.co.jp/books/detail/9784641053854}{出版社サイト}よりデータをダウンロードする.

\begin{Shaded}
\begin{Highlighting}[]
\CommentTok{\# サポートファイルへのリンク}
\NormalTok{curl }\OtherTok{\textless{}{-}} \StringTok{"https://www.yuhikaku.co.jp/static\_files/05385\_support11.zip"}
\CommentTok{\# ダウンロード保存用フォルダが存在しない場合, 作成}
\ControlFlowTok{if}\NormalTok{(}\SpecialCharTok{!}\FunctionTok{dir.exists}\NormalTok{(}\StringTok{"downloads"}\NormalTok{))\{}
    \FunctionTok{dir.create}\NormalTok{(}\StringTok{"downloads"}\NormalTok{)}
\NormalTok{\}}
\NormalTok{cdestfile }\OtherTok{\textless{}{-}} \StringTok{"downloads/support11.zip"}
\FunctionTok{download.file}\NormalTok{(curl, cdestfile)}
\CommentTok{\# データ保存用フォルダが存在しない場合, 作成}
\ControlFlowTok{if}\NormalTok{(}\SpecialCharTok{!}\FunctionTok{dir.exists}\NormalTok{(}\StringTok{"data"}\NormalTok{))\{}
    \FunctionTok{dir.create}\NormalTok{(}\StringTok{"data"}\NormalTok{)}
\NormalTok{\}}
\CommentTok{\# WSL上のRで解凍すると文字化けするので、Linuxのコマンドを外部呼び出し}
\CommentTok{\# Windowsの場合は別途コマンドを用いる.}
\ControlFlowTok{if}\NormalTok{(.Platform}\SpecialCharTok{$}\NormalTok{OS.type }\SpecialCharTok{==} \StringTok{"unix"}\NormalTok{) \{}
    \FunctionTok{system}\NormalTok{(}\FunctionTok{sprintf}\NormalTok{(}\StringTok{\textquotesingle{}unzip {-}n {-}Ocp932 \%s {-}d \%s\textquotesingle{}}\NormalTok{, }\StringTok{"downloads/support11.zip"}\NormalTok{, }\StringTok{"./data"}\NormalTok{))}
\NormalTok{\} }\ControlFlowTok{else}\NormalTok{ \{}
    \FunctionTok{print}\NormalTok{(}\StringTok{"Windowsで解凍するコマンドを別途追加せよ."}\NormalTok{)}
\NormalTok{\}}
\end{Highlighting}
\end{Shaded}

\hypertarget{ux7df4ux7fd2ux554fux984c-11-1-ux78baux8a8d}{%
\section*{練習問題 11-1 {[}確認{]}}\label{ux7df4ux7fd2ux554fux984c-11-1-ux78baux8a8d}}
\addcontentsline{toc}{section}{練習問題 11-1 {[}確認{]}}

\hypertarget{ux7df4ux7fd2ux554fux984c-11-2-ux78baux8a8d}{%
\section*{練習問題 11-2 {[}確認{]}}\label{ux7df4ux7fd2ux554fux984c-11-2-ux78baux8a8d}}
\addcontentsline{toc}{section}{練習問題 11-2 {[}確認{]}}

\hypertarget{ux7df4ux7fd2ux554fux984c-11-3-ux78baux8a8d}{%
\section*{練習問題 11-3 {[}確認{]}}\label{ux7df4ux7fd2ux554fux984c-11-3-ux78baux8a8d}}
\addcontentsline{toc}{section}{練習問題 11-3 {[}確認{]}}

\hypertarget{ux7df4ux7fd2ux554fux984c-11-4-ux78baux8a8d}{%
\section*{練習問題 11-4 {[}確認{]}}\label{ux7df4ux7fd2ux554fux984c-11-4-ux78baux8a8d}}
\addcontentsline{toc}{section}{練習問題 11-4 {[}確認{]}}

\hypertarget{ux7df4ux7fd2ux554fux984c-11-5-ux78baux8a8d}{%
\section*{練習問題 11-5 {[}確認{]}}\label{ux7df4ux7fd2ux554fux984c-11-5-ux78baux8a8d}}
\addcontentsline{toc}{section}{練習問題 11-5 {[}確認{]}}

\hypertarget{ux7df4ux7fd2ux554fux984c-11-6-ux767aux5c55}{%
\section*{練習問題 11-6 {[}発展{]}}\label{ux7df4ux7fd2ux554fux984c-11-6-ux767aux5c55}}
\addcontentsline{toc}{section}{練習問題 11-6 {[}発展{]}}

\hypertarget{ux7df4ux7fd2ux554fux984c-11-7-ux767aux5c55}{%
\section*{練習問題 11-7 {[}発展{]}}\label{ux7df4ux7fd2ux554fux984c-11-7-ux767aux5c55}}
\addcontentsline{toc}{section}{練習問題 11-7 {[}発展{]}}

\hypertarget{ux7df4ux7fd2ux554fux984c-11-8-ux5b9fux8a3c}{%
\section*{練習問題 11-8 {[}実証{]}}\label{ux7df4ux7fd2ux554fux984c-11-8-ux5b9fux8a3c}}
\addcontentsline{toc}{section}{練習問題 11-8 {[}実証{]}}

\hypertarget{ux7df4ux7fd2ux554fux984c-11-9-ux5b9fux8a3c}{%
\section*{練習問題 11-9 {[}実証{]}}\label{ux7df4ux7fd2ux554fux984c-11-9-ux5b9fux8a3c}}
\addcontentsline{toc}{section}{練習問題 11-9 {[}実証{]}}

\hypertarget{ch12}{%
\chapter*{第12章 VARモデル}\label{ch12}}
\addcontentsline{toc}{chapter}{第12章 VARモデル}

先に\href{https://www.yuhikaku.co.jp/books/detail/9784641053854}{出版社サイト}よりデータをダウンロードする.

\begin{Shaded}
\begin{Highlighting}[]
\CommentTok{\# サポートファイルへのリンク}
\NormalTok{curl }\OtherTok{\textless{}{-}} \StringTok{"https://www.yuhikaku.co.jp/static\_files/05385\_support12.zip"}
\CommentTok{\# ダウンロード保存用フォルダが存在しない場合, 作成}
\ControlFlowTok{if}\NormalTok{(}\SpecialCharTok{!}\FunctionTok{dir.exists}\NormalTok{(}\StringTok{"downloads"}\NormalTok{))\{}
    \FunctionTok{dir.create}\NormalTok{(}\StringTok{"downloads"}\NormalTok{)}
\NormalTok{\}}
\NormalTok{cdestfile }\OtherTok{\textless{}{-}} \StringTok{"downloads/support12.zip"}
\FunctionTok{download.file}\NormalTok{(curl, cdestfile)}
\CommentTok{\# データ保存用フォルダが存在しない場合, 作成}
\ControlFlowTok{if}\NormalTok{(}\SpecialCharTok{!}\FunctionTok{dir.exists}\NormalTok{(}\StringTok{"data"}\NormalTok{))\{}
    \FunctionTok{dir.create}\NormalTok{(}\StringTok{"data"}\NormalTok{)}
\NormalTok{\}}
\CommentTok{\# WSL上のRで解凍すると文字化けするので、Linuxのコマンドを外部呼び出し}
\CommentTok{\# Windowsの場合は別途コマンドを用いる.}
\ControlFlowTok{if}\NormalTok{(.Platform}\SpecialCharTok{$}\NormalTok{OS.type }\SpecialCharTok{==} \StringTok{"unix"}\NormalTok{) \{}
    \FunctionTok{system}\NormalTok{(}\FunctionTok{sprintf}\NormalTok{(}\StringTok{\textquotesingle{}unzip {-}n {-}Ocp932 \%s {-}d \%s\textquotesingle{}}\NormalTok{, }\StringTok{"downloads/support12.zip"}\NormalTok{, }\StringTok{"./data"}\NormalTok{))}
\NormalTok{\} }\ControlFlowTok{else}\NormalTok{ \{}
    \FunctionTok{print}\NormalTok{(}\StringTok{"Windowsで解凍するコマンドを別途追加せよ."}\NormalTok{)}
\NormalTok{\}}
\end{Highlighting}
\end{Shaded}

\hypertarget{ux7df4ux7fd2ux554fux984c-12-1-ux78baux8a8d}{%
\section*{練習問題 12-1 {[}確認{]}}\label{ux7df4ux7fd2ux554fux984c-12-1-ux78baux8a8d}}
\addcontentsline{toc}{section}{練習問題 12-1 {[}確認{]}}

\hypertarget{ux7df4ux7fd2ux554fux984c-12-2-ux78baux8a8d}{%
\section*{練習問題 12-2 {[}確認{]}}\label{ux7df4ux7fd2ux554fux984c-12-2-ux78baux8a8d}}
\addcontentsline{toc}{section}{練習問題 12-2 {[}確認{]}}

\hypertarget{ux7df4ux7fd2ux554fux984c-12-3-ux78baux8a8d}{%
\section*{練習問題 12-3 {[}確認{]}}\label{ux7df4ux7fd2ux554fux984c-12-3-ux78baux8a8d}}
\addcontentsline{toc}{section}{練習問題 12-3 {[}確認{]}}

\hypertarget{ux7df4ux7fd2ux554fux984c-12-4-ux767aux5c55}{%
\section*{練習問題 12-4 {[}発展{]}}\label{ux7df4ux7fd2ux554fux984c-12-4-ux767aux5c55}}
\addcontentsline{toc}{section}{練習問題 12-4 {[}発展{]}}

\hypertarget{ux7df4ux7fd2ux554fux984c-12-5-ux5b9fux8a3c}{%
\section*{練習問題 12-5 {[}実証{]}}\label{ux7df4ux7fd2ux554fux984c-12-5-ux5b9fux8a3c}}
\addcontentsline{toc}{section}{練習問題 12-5 {[}実証{]}}

\hypertarget{ux7df4ux7fd2ux554fux984c-12-6-ux5b9fux8a3c}{%
\section*{練習問題 12-6 {[}実証{]}}\label{ux7df4ux7fd2ux554fux984c-12-6-ux5b9fux8a3c}}
\addcontentsline{toc}{section}{練習問題 12-6 {[}実証{]}}

  \bibliography{book.bib,packages.bib}

\end{document}
